\section{Diseño Físico}

\noindent Para la implementación del diseño físico de la solución se ha utilizado el motor de base de datos Microsoft SQL Server 2012 (Express Edition). Esta edición es gratuita, de libre descarga aunque tiene ciertas restricciones (en cuanto a la utilización de recursos de hardware y a las herramientas disponibles). Sin embargo, para la implementación requerida para este trabajo práctico, es mucho más que suficiente.\\
Para la administración y, manipulación de datos y estructura de la base, la herramienta elegida fue el Microsoft SQL Server 2012 Management Studio (MSQLS2012SMS), el cual también es de libre descarga y ha permitido llevar a cabo las tareas derivadas del trabajo práctico con mucha comodidad.\\
Para definición de las tablas y restricciones nos hemos valido de las herramientas gráficas provistas por MSQLS2012MS.
Con respecto a las decisiones tomadas para la implementación, podemos enumerar las siguientes:\\
\begin{enumerate}
\item Todos los ids se han implementado con tipo int\\
\item Los campos numéricos se han implementado con tipo int\\
\item Los campos de texto utilizados como nombres o breves descripciones se han implementado con tipo nvarchar(50)\\
\item Los campos de texto de mayor extensión (testimonios, descripciones largas, etc) se han implementado con tipo nvarchar(MAX).\\
\item Se han unificado aquellos campos con fecha y hora, implementando este con el tipo datetime.\\
\item Para aquellos campos de tipo fecha, donde no tiene relevancia la hora, se ha utilizado el tipo date.\\
\end{enumerate}

\subsection{Implementación de restricciones}

\noindent Las restricciones se han implementado como constrains o bien como triggers que se disparan luego de realizar un insert o update de alguna tabla involucrada.\\

\begin{itemize}

\item La fecha de ingreso de un caso debe ser posterior a la fecha de ocurrencia.\\
\begin{lstlisting}
ALTER TABLE CASO ADD  CONSTRAINT CK_CASO_FechaOcurrenciaMenorIgualFechaIngreso CHECK  (([fecha_ocurrencia]<=[fecha_ingreso]))
\end{lstlisting}

\item La fecha de ingreso de la evidencia debe ser posterior a la fecha de recolección y de sellado.\\
\begin{lstlisting}
ALTER TABLE EVIDENCIA ADD  CONSTRAINT CK_EVIDENCIA_VALIDAR_FECHAS_SELLADO_HALLAZGO_INGRESO CHECK  (([fecha_hallazgo]<=[fecha_sellado] AND [fecha_sellado]<=[fecha_ingreso]))
\end{lstlisting}

\item La fecha de recolección de evidencia debe ser posterior a la fecha de ingreso del caso correspondiente.\\
\begin{lstlisting}
CREATE TRIGGER ValidarFechasEvidencia
   ON  EVIDENCIA
   FOR INSERT, UPDATE
AS 
BEGIN	
	DECLARE @fechaRecoleccion as Datetime
	DECLARE @fechaCaso as datetime	
	DECLARE @id_caso as int

    SELECT @fechaRecoleccion = fecha_hallazgo, @id_caso = id_caso
	FROM inserted

	SELECT @fechaCaso = fecha_ocurrencia
	FROM CASO
	WHERE id_caso = @id_caso

	IF @fechaRecoleccion < @fechaCaso

	BEGIN
		RAISERROR('No se puede tener una evidencia con fecha de hallazgo menor a la de la ocurrencia del caso', 10,1);
		ROLLBACK TRANSACTION
	END
END
\end{lstlisting}

\item La fecha de ocurrencia de un evento debe ser anterior a la fecha de ingreso del caso correspondiente.\\
\begin{lstlisting}
CREATE TRIGGER ValidarFechasEvento
   ON  EVENTO
   FOR INSERT, UPDATE
AS 
BEGIN	
	DECLARE @fechaEvento as Datetime
	DECLARE @fechaCaso as datetime	
	DECLARE @id_caso as int

    SELECT @fechaEvento = fecha_evento, @id_caso = S.id_caso
	FROM inserted I
	INNER JOIN SUCESO S
		ON I.id_evento = S.id_evento


	SELECT @fechaCaso = fecha_ocurrencia
	FROM CASO
	WHERE id_caso = @id_caso

	IF @fechaEvento < @fechaCaso

	BEGIN
		RAISERROR('No se puede tener un evento con fecha menor a la de la ocurrencia del caso', 10,1);
		ROLLBACK TRANSACTION
	END
END
\end{lstlisting}

\item Un investigador principal no puede ser auxiliar del mismo caso\\
\begin{lstlisting}
CREATE TRIGGER ValidarAuxiliaresPrincipal
   ON  AUXILIAR
   FOR INSERT, UPDATE
AS 
BEGIN
	DECLARE @dni_oficial as int
	DECLARE @id_caso as int	
	DECLARE @cantidad as int
	

    SELECT @dni_oficial = dni_oficial, @id_caso = id_caso
	FROM inserted


	SELECT @cantidad = COUNT(*) 
	FROM CASO
	WHERE id_caso = @id_caso AND dni_oficial = @dni_oficial

	IF @cantidad > 0

	BEGIN
		RAISERROR('No se puede tener un auxiliar que sea principal del mismo caso', 10,1);
		ROLLBACK TRANSACTION
	END
END
\end{lstlisting}

\item Un investigador auxiliar no puede ser principal del mismo caso\\
\begin{lstlisting}
CREATE TRIGGER ValidarPrincipalAuxiliares
   ON  CASO
   FOR INSERT, UPDATE
AS 
BEGIN
	DECLARE @dni_oficial as int
	DECLARE @id_caso as int	
	DECLARE @cantidad as int
	

    SELECT @dni_oficial = dni_oficial, @id_caso = id_caso
	FROM inserted


	SELECT @cantidad = COUNT(*) 
	FROM AUXILIAR
	WHERE id_caso = @id_caso AND dni_oficial = @dni_oficial

	IF @cantidad > 0

	BEGIN
		RAISERROR('No se puede tener un auxiliar que sea principal del mismo caso', 10,1);
		ROLLBACK TRANSACTION
	END
END
\end{lstlisting}

\item Una persona culpable debe estar involucrado en el caso\\
\begin{lstlisting}
CREATE TRIGGER CulpableDebeSerInvolucrado
   ON  CULPABLE
   FOR INSERT, UPDATE
AS 
BEGIN
	DECLARE @dni as int
	DECLARE @id_caso as int	
	DECLARE @cantidad as int
	

    SELECT @dni = I.dni, @id_caso = E.id_caso
	FROM inserted I
	INNER JOIN ESTADO E
		ON I.id_estado = E.id_estado


	SELECT @cantidad = COUNT(*) 
	FROM CASO C
	INNER JOIN INVOLUCRADO I
		ON C.id_caso = I.dni
	WHERE C.id_caso = @id_caso AND I.dni = @dni

	IF @cantidad = 0
	BEGIN
		RAISERROR('No se puede tener un culpable que no sea un involucrado del caso', 10,1);
		ROLLBACK TRANSACTION
	END
END
\end{lstlisting}

\item Los sucesos están asociados únicamente a personas involucradas\\
\begin{lstlisting}
CREATE TRIGGER LosSucesosDebenSerDeInvolucrados
   ON  SUCESO
   FOR INSERT, UPDATE
AS 
BEGIN
	DECLARE @dni as int
	DECLARE @id_caso as int	
	DECLARE @cantidad as int
	

    SELECT @dni = dni, @id_caso = id_caso
	FROM inserted

	SELECT @cantidad = COUNT(*) 
	FROM CASO C
	INNER JOIN INVOLUCRADO I
		ON C.id_caso = I.dni
	WHERE C.id_caso = @id_caso AND I.dni = @dni

	IF @cantidad = 0
	BEGIN
		RAISERROR('No se puede tener un suceso que no sea llevado a cabo a un involucrado al caso', 10,1);
		ROLLBACK TRANSACTION
	END
END
\end{lstlisting}

\item No puede haber una persona involucrada al caso que sea oficial principal o auxiliar del mismo\\
\begin{lstlisting}
CREATE TRIGGER LosInvolucradosNoPuedenSerOficialesDelCaso
   ON  INVOLUCRADO
   FOR INSERT, UPDATE
AS 
BEGIN
	DECLARE @dni as int
	DECLARE @id_caso as int	
	DECLARE @cantidadPrincipal as int
	DECLARE @cantidadAuxiliar as int
	

    SELECT @dni = dni, @id_caso = id_caso
	FROM inserted

	SELECT @cantidadPrincipal = COUNT(*) 
	FROM CASO C

	SELECT @cantidadAuxiliar = COUNT(*)
	FROM AUXILIAR A
	WHERE A.dni_oficial = @dni AND A.id_caso = @id_caso


	IF @cantidadAuxiliar + @cantidadPrincipal > 0
	BEGIN
		RAISERROR('No se puede tener un involucrado que sea oficial principal o auxiliar del caso', 10,1);
		ROLLBACK TRANSACTION
	END
END
\end{lstlisting}
\end{itemize}
