\section{Conclusiones}

\noindent A través de este trabajo se observó es experimento las ventajas del uso del modelado en el proceso de la creación de una bases de datos para la resolución de problemas.\\\\
Gracias al uso del modelo de entidad relación fuimos modificando las ideas iniciales que teníamos para la solución del problem encontramos problemas que inicialmente no imaginamos y encontramos formas que visualmente observamos que facilitaban las consultas que debimos resolver.\\\\
El modelo relacional logro que el traspaso de DER a la creacion de su base de datos sea practicamentre trivial. El uso de estos modelos nos ahorro los problemas que hubieran ocurrido si solo nos volcaramos al la implementación directa, como el tener que volver a insertar datos por cada cambio en la cantidad y forma de las tablas de la base o modificación excesiva de toda la base de datos al encontrar problemas para poder hacer las consultas pedidas.\\\\
El modelos de entidad relación y relacional puede reducir significamente los problemas de diseño de una base de datos, ahorrando así horas de trabajo hombre. El uso de modelado, creemos, es recomendable y extremadamente útil. 
